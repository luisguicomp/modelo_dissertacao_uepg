\chapter{REVISÃO LITERATURA}

A Bioinformática, também conhecida como Biologia Computacional, consiste na aplicação da informática para resolver problemas biológicos \cite{gibas2001developing}.

\section{EXEMPLO DE SEÇÃO}

A seguir, temos um exemplo de tabela

\begin{table}[!ht]
\caption{Exemplo de Tabela}
\label{ex_tabela}
\begin{tabularx}{\linewidth}{l c X c}
\hline
\multicolumn{1}{l}{\textbf{Experimento}} &
\multicolumn{1}{c}{\textbf{Protocolo}} &
\multicolumn{1}{l}{\textbf{Métrica}} &
\multicolumn{1}{c}{\textbf{Equação}} \\ \hline

Exemplo 1 & 1 & Acurácia Média & 1\\
\hline
Exemplo 2 & 1 & Acurácia Balanceada & 2 \\
\hline
Exemplo 3 & 1 & Média Geométrica & 3 \\
\hline
Exemplo 4 & 2 & \emph{Overfitting} (AM) & 1 e 4 \\
\hline
Exemplo 5 & 2 & \emph{Overfitting} (AB) & 2 e 4 \\
\hline
Exemplo 6 & 3 & Ranqueamento \emph{Relief} & 5 \\ 
\hline
Exemplo 7 & 1 & \emph{AM, AB, OVAM e OVAB} & 1,2 e 4\\ 
\hline
\text{\footnotesize Fonte: O autor.}
\end{tabularx}
\end{table}

\subsection{Exemplo de Subseção}


